\chapter{Potvrzení simulace}
\label{chap:sim_confirmation}

Po navržení a implementaci simulace je dalším krokem sledování její přesnosti a porovnání s reálnými experimenty. První způsob porovnání již byl zmíněn v \autoref{chap:drag}, kde jsme porovnávali simulované a reálné dopady tření na spinner a to bez externího magnetického pole (viz \autoref{fig:drag_fit}, \autoref{fig:drag_fit_wlin}) i v externím magnetickém poli (viz \autoref{fig:drag_fit_mag}, \autoref{fig:drag_fit_mag_wlin}). Všechna tato měření byla prozatím velmi podobná našim simulovaným hodnotám, avšak použit byl pouze jeden spinner. Další experimenty se tedy zaměří na další okrajové případy a použití více spinnerů.

\section{Vysokorychlostní videozáznam}

Nevýhodou minulého měření pomocí magnetického čidla Vernier je, že jsme nebyli schopni přesně určit okamžitou pozici, či rychlost. Toto nebyl tak velký problém, jelikož náš celkový záznam byl velmi dlouhý a nezajímaly nás mikroskopická chování spinneru, ale spíše makroskopický dopad tření na jeho schování. Nyní však budeme sledovat mikroskopické chování a to trackováním vyznačeného bodu na spinneru natočeného vysokorychlostní kamerou. K trackování použijeme volně dostupný software zvaný \texttt{Tracker} \cite{Tracker}. Snímkovou frekvenci jsme nastavili v jednom experimnetu na 60 snímků za sekundu, ve zbylých experimentech na 1000 snímků za sekundu.

\subsection{Aparatura}