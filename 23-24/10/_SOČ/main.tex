% ŠABLONA PRO PSANÍ SOČ
%%%%%%%%%%%%%%%%%%%%%%%%%
% Autor: Jakub Dokulil (kubadokulil99@gmail.com)
% Tato šablona byla vytvořena tak, aby pomocí ní mohli v systému LaTeX soutěžící sázet své práce a zároveň odpovídala požadavkům na formátování vyplývajícím z wordové šablony umístěné na webu soc.cz.
%

% !TEX root = ./main.tex

\documentclass[12pt, a4paper,
  %oneside,      %% -- odkomentujte, pokud chcete svou práci mít pouze jednostrannou, mezera pro hřbet pak automaticky bude pouze na levé straně
 twoside,        %% -- pro oboustranné práce, mezera pro hřbet následně střídá strany.
 openright
]{report}

%% Nutné balíčky a nastavení
%%%%%%%%%%%%%%%%%%%%%%%%%%%%

%% Proměnné
\newcommand\city{Praha} %% vyplň oficiální název města
\newcommand\district{Hlavní město Praha} %% vyplň oficiální název kraje
\newcommand\specialization{Obor č. 2: Fyzika} %% -- napiš číslo a název tvého oboru
\newcommand\school{Gymnázium Christiana Dopplera} %% vyplň název školy
\newcommand\schoolAddress{Zborovská 621, 150 00 Malá Strana} %% vyplň název školy
\newcommand\consultant{RNDr. Pavel Josef, CSc.} %% vyplň jméno svého konzultanta
\newcommand\authorName{Ondřej Sedláček}  %% vyplň své jméno
\newcommand\publicationYear{2023} %% vyplň rok
\newcommand\mainTitle{Počítačové modelování dynamických magnetických systémů} %% vyplň název své práce
\newcommand\mainTitleEN{Computer modelling of dynamic magnetic systems} %% vyplň název své práce

\title{\mainTitle} %% -- Název tvé práce
\author{\authorName} %% -- tvé jméno
\date{\publicationYear} %% -- rok, kdy píšeš SOČku

\usepackage[top=2.5cm, bottom=2.5cm, left=3.5cm, right=1.5cm]{geometry} %% nastaví okraje, left -- vnitřní okraj, right -- vnější okraj

\usepackage[czech]{babel} %% balík babel pro sazbu v češtině
\usepackage[utf8]{inputenc} %% balíky pro kódování textu
\usepackage[T1]{fontenc}
\usepackage{cmap} %% balíček zajišťující, že vytvořené PDF bude prohledávatelné a kopírovatelné

\usepackage{graphicx} %% balík pro vkládání obrázků

\usepackage{subcaption} %% balíček pro vkládání podobrázků
\usepackage{wrapfig}
\usepackage{hyperref} %% balíček, který v PDF vytváří odkazy

\linespread{1.15} %% řádkování

\usepackage[pagestyles]{titlesec} %% balíček pro úpravu stylu kapitol a sekcí
\titleformat{\chapter}[block]{\scshape\bfseries\LARGE}{\thechapter}{10pt}{\vspace{0pt}}[\vspace{-22pt}]
\titleformat{\section}[block]{\scshape\bfseries\Large}{\thesection}{10pt}{\vspace{0pt}}
\titleformat{\subsection}[block]{\bfseries\large}{\thesubsection}{10pt}{\vspace{0pt}}

\setcounter{secnumdepth}{2}
\setcounter{tocdepth}{4}
\usepackage{fancyhdr}
\pagestyle{fancy}
\renewcommand{\headrulewidth}{1pt}

\addto\captionsczech{\renewcommand{\figurename}{Obr.}}
\addto\captionsczech{\renewcommand{\tablename}{Tab.}}
\counterwithout{figure}{chapter}
\counterwithout{table}{chapter}
\counterwithout{equation}{chapter}
\counterwithout{footnote}{chapter}

\setlength{\parskip}{12pt}%
\setlength{\parindent}{0pt}%

% \usepackage[]{nomencl}
% % makeindex main.nlo -s nomencl.ist -o main.nls
\usepackage{tabularx} % in the preamble

\usepackage{booktabs}

\usepackage{url}

%% Balíčky co se můžou hodit :) 
%%%%%%%%%%%%%%%%%%%%%%%%%%%%%%%

\usepackage{pdfpages} %% Balíček umožňující vkládat stránky z PDF souborů, 

\usepackage{upgreek} %% Balíček pro sazbu stojatých řeckých písmen, třeba u jednotky mikrometr. Například stojaté mí: \upmu, stojaté pí: \uppi

\usepackage{amsmath}    %% Balíčky amsmath a amsfonts 
\usepackage{yhmath}    %% Balíčky amsmath a amsfonts 
\usepackage{amsfonts}   %% pro sazbu matematických symbolů
\usepackage{esint}     %% pro sazbu různých integrálů (např \oiint)
\usepackage{mathrsfs}
\usepackage{enumitem}

%% makra pro sazbu matematiky
\newcommand{\dif}{\mathrm{d}} %% makro pro sazbu diferenciálu, místo toho
%% abych musel psát '\mathrm{d}' mi stačí napsat '\dif' což je mnohem 
%% kratší a mohu si tak usnadnit práci

%% Bordel pro práci - můžeš smáznout :) 
%%%%%%%%%%%%%%%%%%%

\usepackage{lipsum} %% balíček který píše lipsum (nesmyslný text, který se používá pro kontrolu typografie)

%% Začátek dokumentu
%%%%%%%%%%%%%%%%%%%%

\begin{document}
\pagestyle{empty}
\pagenumbering{Roman}

% PŘEDNÍ STRANA
\begin{titlepage}
    \bfseries{ %%% písmo na stránce je tučně
        \begin{center}
            \LARGE{STŘEDOŠKOLSKÁ ODBORNÁ ČINNOST}

            \vspace{14pt}
            \large{ %%%%
                \specialization
            } %%%%

            \vspace{0.3 \textheight}

            \LARGE{ %%%%
                \mainTitle
            }%%%%

            \vspace{0.35 \textheight}
        \end{center}

        \noindent\Large{\authorName}

        \noindent\Large{\district \hspace{\stretch{1}}  \city, \publicationYear}


    } %%%
\end{titlepage}
\cleardoublepage

% ÚVODNÍ STRANA
% \begin{}
%% Úvodní stránka s informacemi
{\bfseries %%% písmo na stránce je tučně
    \begin{center}
        \LARGE{STŘEDOŠKOLSKÁ ODBORNÁ ČINNOST}

        \vspace{14pt}
        {\large %%%%
            \specialization %% -- napiš číslo a název tvého oboru
        } %%%%

        \vspace{0.20 \textheight}

        \LARGE{ %%%%
            \mainTitle
        }

        \LARGE{ %%%%
            \mainTitleEN
        }%%%%

        \vspace{0.25\textheight}
    \end{center}
}%%%
{\Large %%%
    \noindent\textbf{Jméno:} \authorName\\
    \textbf{Škola:} \school, \schoolAddress\\
    \textbf{Kraj:} \district\\
    \textbf{Konzultant:} \consultant\\
} %%%

\noindent \city, \publicationYear
% \end{}
\cleardoublepage

% PROHLÁŠENÍ
% \begin{}
\noindent{\Large{\bfseries{Prohlášení}}}

\noindent Prohlašuji, že jsem svou práci SOČ vypracoval samostatně a použil jsem pouze prameny a literaturu uvedené v seznamu bibliografických záznamů.

\noindent Prohlašuji, že tištěná verze a elektronická verze soutěžní práce SOČ jsou shodné.

\noindent Nemám závažný důvod proti zpřístupňování této práce v souladu se zákonem č. 121/2000 Sb., o právu autorském, o právech souvisejících s právem autorským a o změně některých zákonů (autorský zákon) ve znění pozdějších předpisů.

\vspace{24 pt}

\noindent V Praze dne 9. září 2023 \dotfill{}\hspace{\stretch{0.9}}

\hspace{6cm} \authorName
% \end{}
\cleardoublepage

% PODĚKOVÁNÍ TODO
% \begin{}
\vspace*{0.0\textheight}
% \vspace*{0.8\textheight}
\noindent{\Large{\bfseries{Poděkování}}}

\noindent
Chtěl bych poděkovat ...
% \end{}
\cleardoublepage

% ANOTACE TODO
% \begin{}
\noindent{\Large{\bfseries{Anotace}}}

\noindent Sem napíšeš svůj abstrakt. \lipsum[1] %% přepiš!!

\vspace{18pt}

% TODO - keywords
\noindent{\Large{\bfseries{Klíčová slova}}}

\noindent Šablona, \LaTeX, SOČ, \dots

\vspace{18pt}

% TODO - abstrakt EN
\noindent{\Large{\bfseries{Annotation}}}

\noindent Write your abstract here! \lipsum[1] %% přepiš!!

\vspace{18pt}

% TODO - keywords EN
\noindent{\Large{\bfseries{Keywords}}}

\noindent Template, \LaTeX, High school proffessional activity, \dots
% \end{}
\cleardoublepage

% TABLE OF CONTENTS
% \begin{}
\setlength{\parskip}{0pt}%
\tableofcontents
\setlength{\parskip}{12pt}%

\pagenumbering{arabic}
\pagestyle{fancy}
\setcounter{page}{1}
% \end{}

\chapter{Úvod}
\label{chap:introduction}
Motivací pro tuto práci byla úloha mezinárodní fyzikální soutěže zvané \textit{"International Young Physicists Tournament"}, neboli IYPT.
U nás je avšak tato soutěž známější pod zkratkou TMF vycházející z překladu původního názvu - \textit{"Turnaj mladých fyziků"}.
Soutěž se v České republice kéná pod záštitout Fakulty jaderné a fyzikálně inženýrské ČVUT, FZU AV ČR, MŠMT a JČMT a z podstaty soutěže je cílem úloh dovést středoškolské studenty k vědeckému sledování nejrůznějších jevů ze všech částí fyziky.

Úloha, kterou jsem se zabýval, je v pořadí desáta úloha letošního, tedy 37., ročníku. Zadání je následovné \cite{tmf_tasks}:

\textbf{10. Magnetický převod}

\textit{"Vezměte několik identických prstových točítek \footnote{Z překladu anglického "Fidget spinner"} a připevněte k jejich koncům neodymové magnety. Pokud umístíte točítka v rovině vedle sebe a točíte jedním z nich, ostatní se začnou otáčet jen vlivem magnetického pole. Prozkoumejte a vysvětlete tento jev."}

Zadání úlohy je, jak je pro TMF tradiční, velmi otevřené a je tedy na řešiteli, aby si vymezil přesnou oblast svého zkoumání.
Tato práce se bude zabívat:

\begin{enumerate}[topsep=0pt, partopsep=0pt]
    \setlength{\itemsep}{0pt}%
    \setlength{\parskip}{0pt}%
    \item Určením vlastností \textit{prstových točítek} (dále "fidget spinner" či pouze "spinner")
    \item Popisem třecích sil působících na spinner
    \item Vývojem simulace chování systémů více fidget spinnerů a porovnáním této simulace s realitou
    \item Přenosem úhlové rychlosti
    \item Přenosem momentu síly
    \item Možným využím získaných poznatků k vývoji efektivnějších magnetických převodů
\end{enumerate}

\clearpage

\section[Nomenklatura]{Nomenklatura}
\label{sec:nomenclature}
V tabulce \ref{tab:nomenclature} definujeme symboly, které budeme používat v průběhu celé práce, společně s jejich významem:

\begin{table}[!ht]
    \captionsetup{justification=raggedright,singlelinecheck=off}
    \caption{Nomenklatura}
    \label{tab:nomenclature}

    \centering{\textbf{Vlastnosti spinneru}} \\
    \begin{tabularx}{\textwidth}{m{0.1\textwidth} m{0.1\textwidth} p{0.4\textwidth} p{5cm} }
        \textbf{Symbol}  & \textbf{Jednotka}         & \textbf{Popis}                                                & \textbf{Poznámka}                              \\
        \hline
        $n$              &                           & Celkový počet ramen                                           & Ekvivaletní počtu magnetů                      \\
        $r$              & $m$                       & Poloměr spinneru                                              & Ekvivaletní vzdálenosti osy otáčení od magnetů \\
        $S$              & $(m,m,0)$                 & Střed spinneru v rovině                                       & (tzn. pozice osy otáčení)                      \\
        $P(i)$           & $(m,m,0)$                 & Pozice $i$. magnetu spinneru                                  & Funkce indexu magnetu                          \\
        $\varphi$        & $\text{rad}$              & Úhel rotace spinneru                                                                                           \\
        $\omega$         & $\text{rad} \cdot s^{-1}$ & Úhlová rychlost spinneru                                                                                       \\
        $I$              & $kg \cdot m^2$            & Moment setrvačnosti spinneru                                                                                   \\
        $\alpha$         & $\text{rad} \cdot s^{-2}$ & Koeficient rychlostně nezávislého brzdného úhlového zrychlení & viz \ref{}                                     \\
        $\beta$          & $s^{-1}$                  & Koeficient lineárně závislého brzdného úhlového zrychlení     & viz \ref{}                                     \\
        $\gamma$         & $\text{rad}^{-1}$         & Koeficient kvadraticky závislého brzdného úhlového zrychlení  & viz \ref{}                                     \\
        $c_1$, $t_{max}$ & $s$                       & Celková délka otáčení spinneru                                & viz \ref{}                                     \\
    \end{tabularx}

    \vspace{24pt}

    \centering{\textbf{Vlastnosti magnetu}} \\
    \begin{tabularx}{\textwidth}{m{0.15\textwidth} m{0.1\textwidth} p{0.45\textwidth} p{4cm} }
        \textbf{Symbol}                      & \textbf{Jednotka}                                                            & \textbf{Popis}                                                                              & \textbf{Poznámka}                      \\
        \hline
        $\vec{m}$                            & $A \cdot m^2$                                                                & Magnetický moment                                                                                                                    \\
        $\vec{B}_r$                          & $T$                                                                          & Remanentní magentizace                                                                      & (tzv. remanence)                       \\
        $V$                                  & $m^3$                                                                        & Objem magnetu                                                                                                                        \\
        $F_m(\vec{r}, \vec{m_1}, \vec{m_2})$ & $N$                                                                          & Silová interakce mezi magnetickými momenty $\vec{m_1}$ a $\vec{m_2}$ vzdálenými o $\vec{r}$ & \cite{magnetic_force}                  \\
        $B(\vec{r}, \vec{m})$                & $T$                                                                          & Magnetická indukce tvořená magnetickými momentem $\vec{m}$ ve vzdálenosti $\vec{r}$         & \cite{magnetic_force, magnetic_torque} \\
        $\tau_F, \tau_{mag}$                 & $Nm$
                                             & Momenty sil působící na spinner vycházející ze silové a magnetické interakce & \cite{magnetic_torque,torque_def}, viz \ref{}                                                                                        \\
    \end{tabularx}
\end{table}

Dále stojí za zmínku, že pro vyjádření chyby měření je v textu používána tzv. \textit{shorthand error notation} \cite{shorthand_error_notation} (pro naše účely zkráceno na SEN). Pro jasnost uvedeme příklad, kde zápis pomocí SEN vypadá takto: $11.5(12)$, a ekvivalentní přepis do standardní notace je: $11.5 \pm 1.2$. Tímto zjednodušíme zápis: $11.5 \pm 1.2 = 11.5(12)$ \cite{shorthand_error_notation_stack_exchange}.

\chapter{Úvodní sledování}
\begin{wrapfigure}{r}{0.4\textwidth}
    \includegraphics[width=0.3\textwidth]{imgs/osazeny_spinner.png}
    \centering
    \caption{Spinner osazeným neodymovými magnety}
    \label{fig:1spinner}
\end{wrapfigure}

Prvním krokem v řešení této úlohy bylo kvalitativní sledování jejich chování v co největším rozpětí konfigurací, abychom mohli určit relevantní parametry a odstranit nezajímavé konfigurace.

Hlavním poznatkem je změna chování systému dvou pinnerů v závislosti na jejich relativních rychlostech.
Máme-li na stole 2 spinnery, ze kterých je jeden nehybný, a druhý roztočíme na nízké otáčky, dojde po krátké chvíli k silné, ale chaotické, interakci (viz příloha \ref{attachment_1}).
Naopak, roztočíme-li druhý spinner znatelně rychleji, nedochází téměř k žádné interakci (viz příloha \ref{attachment_2}).
Druhý spinner se prvnímu spinneru efektivně jeví jako permanentí magnet - druhý magnet se tedy nanejvýše umístí do energeticky nejvýhodnější polohy a dále zůstává nehybný.

\begin{wrapfigure}{l}{0.45\textwidth}
    \includegraphics[width=0.35\textwidth]{imgs/3spinnery.png}
    \centering
    \caption{Tři interagující spinnery}
    \label{fig:3spinners}
\end{wrapfigure}
Dalším cenným poznatkem je, že při interakci více spinnerů se systém chová chaoticky téměř vždy (viz příloha \ref{attachment_3}).
Toto pro nás dělá měření interakcí více jak dvou spinnerů nepříznivé a k získání použitelných výsledků je důležité omezit naše bádání pouze na jeden či dva spinnery.
Poté, co kvalitně popíšeme menší počet spinnerů, se můžeme pomocí simulace pokusit o extrapolování našeho modelu na více spinnerů.

Z přesného zadání můžeme také vyčíst nějaké důležité předpoklady.
Jmenovitě se jedná o
umístění všech spinnerů v jedné \textit{"rovině vedle sebe"}, což z velké míry usnadní budoucí výpočty,
točením pouze \textit{"jedním z nich"} a
omezení interakcí mezi spinnery pouze na \textit{"vlivy magentického pole"}.
Neměli bychom opomenout ani skutečnost, že všechny spinnery mají být \textit{"identické"}.

\section{Určení parametrů}

Z úvodního sledování není těžké určit relevantní parametry a vybrat, které z nich je možné s naším vybavením měřit.

\subsection{Parametry týkající se konfigurace spinnerů}
Jakožto nejdůležitější bychom určitě označili relativní pozice všech spinnerů, které popíšeme pomocí jejich středů $S_1, S_2,...$ (každý střed je braný jako vektor ve spinnerové rovině) a jejich poloměrů $r_1, r_2, ...$.
Dále bude k přesnému určení pozicí magnetů nezbytné znát okamžité úhly rotace spinnerů, které ozačíme $\varphi_1, \varphi_2,...$.
Nakonec k popsání spinneru musíme určit počet ramen, neboli počet připevněných magnetů. Tento počet označíme $n$ a pro naše spinnery platí $n=3$.

Pomocí těchto údajů je triviální vyjádřit pozici $P(i)$ libovolného magnetu pomocí jeho indexu $i$ (kde $0 \leq i < n$):

\begin{equation}
    \label{eq:magnet_pos}
    P(i) = S + \biggr(r\cos{\bigg(\varphi + \frac{2\pi i}{n}\bigg)},
    r\sin{\bigg(\varphi+\frac{2\pi i}{n}}\bigg), 0 \biggr)
\end{equation}

\begin{wrapfigure}{r}{0.4\textwidth}
    \includegraphics[width=0.3\textwidth]{imgs/orientace_magnetu.png}
    \centering
    \caption{Tři námi vyhranění orientace magnetů}
    \label{fig:mag_orientations}
\end{wrapfigure}

Kromě pozice magnetu hraje také klíčovou roli nasměrování jeho pólů. Zde definujeme 3 důležité orientace pólů (viz Obr. \ref{fig:mag_orientations}):

\begin{enumerate}[topsep=0pt, partopsep=0pt]
    \setlength{\itemsep}{0pt}%
    \setlength{\parskip}{0pt}%
    \item \textbf{Vertikální (\textit{vertical})}
    \item Odstředivá (\textit{eccentric})
    \item Tečná (\textit{tangent})
\end{enumerate}

V našich experimentech jsme používali převážně vertikální konfiguraci, jelikož takto bylo uchycení magnetů nejjednodušší.
Magnety se totiž samy připevnily ke kovovému závaží v každém z ramen spinneru (viz Obr. \ref{fig:1spinner}), které je zde z důvodu zvýšení momentu setrvačnosti.

Tyto konfigurace jsme také schopni popsat a to opět jakožto funkci indexu magnetu.
Nejdříve vyjádříme směrový vektor $\vec{u}(i)$ pro $i$. magnet v každé konfiguraci:

\vspace{24pt}

\begin{table}[!ht]
    \captionsetup{justification=raggedright,singlelinecheck=off}
    \caption{Směrové vektory pro různé konfigurace}
    \label{tab:mag_dir_vec}

    \def\arraystretch{1.5}
    \begin{tabularx}{\textwidth}{p{0.50\textwidth} p{0.50\textwidth} }
        \textbf{Konfigurace} & \textbf{Směrový vektor magnetu}                    \\
        \hline
        Vertikální           & $\vec{u}(i) = (0,0,1)$                             \\
        Odstředivá           & $\vec{u}(i) = \widehat{P(i) - S}$                  \\
        Tečná                & $\vec{u}(i) = (0,0,1) \times \widehat{(P(i) - S)}$ \\
    \end{tabularx}
\end{table}

{\raggedright
Když nyní zvolíme velikost magnetického momentu našich magnetů $|\vec{m}_0|$, jsme schopni popsat magnetický moment včetně jeho velikosti \footnote{Všimněme si, že $u(i) = \hat{u}(i)$}:}

\begin{equation}
    \label{eq:magnet_moment_orientation}
    \vec{m} = |\vec{m}_0| \cdot \vec{u}(i)
\end{equation}

Velikost magentických momentů bude záviset na teplotě, velikosti a materiálových vlastnostech magnetů (např. jejich chemickém složení a kvalitě), ale to, jaká je pravá velikost magnetických momentů $|\vec{m}_0|$, je momentálně nepodstatné a určíme ji později (viz kap. \ref{}).

Posledním parametrem, který zmíníme, ale nebudeme se jím zabývat, je přitahování, či odpuzování magnetů.
V našem případě jsem se zaměřili převážně na systémy, kde se všechny magnety odpuzují.

\subsection{Parametry týkající se pohybu spinnerů}

Druhou, složitější, částí popisu našeho systému je jeho pohyb a chování v čase.
Zde se nevyhneme úhlovým rychlostem jednotlivých spinnerů, které budeme značit $\omega_1, \omega_2,...$.
Poté by nás přirozeně napadlo úhlové zrychlení $\alpha_1, \alpha_2, ...$, ale pro náš případ bude šikovnější využít toho, že $\tau = I\alpha$, kde $\tau$ značí moment síly a $I$ značí moment setrvačnosti\footnote{Někdy také značeno $J$} spinneru.
Moment setrvačnosti určíme později ve své vlastní kapitole (viz kap. \ref{}).

Posledním parametrem, který zmíníme, je tření v ložiscích spinnerů a jiné odporové síly.
Těm se budeme do hloubky věnovat v kapitole \ref{}.

\clearpage

\section{Modelování magnetů}
V průběhu našich experimentů používáme neodymové (NdFeB) magnety krychlového tvaru o hraně 5mm a jakosti N35. \footnote{Jakosti neodymových magnetů se pohybují od N35 do N55 \cite{magnet_grades}.}
Toto označení jakosti neodymových magnetů popisuje jejich chemické složení, tepelnou odolonost a hlavně sílu \cite{magnet_grades}, která je popsána pomocí tzv. remanentní magnetizace, neboli remanence.

Jelikož jsou magnety poměrně malé, můžeme je ve větších vzdálenostech aproximovat jakožto magnetické dipóly.
Zároveň existuje velmi elegantní způsob, jak vypočítat velikost magnetického dipólu z jeho remanence \cite{magnetic_torque}:

\begin{equation}
    \label{eq:mag_mom_remanence}
    |\vec{m}_0| = \frac{1}{\mu_0}|\vec{B}_r|V
\end{equation}

Tabulkové hodnoty pro remanenci NdFeB magnetů jsou sice známé, ale v našem případě budeme přesnou hodnotu $|\vec{B}_r|$ našich magnetů měřit později, v kapitole \ref{}.

\section{Popis magnetických interakcí}

Jelikož k popisu magnetů používáme idealizaci pomocí magnetických dipólů, můžeme popsat interakce mezi nimi pomocí následujících rovnic.

1. Silové interakce \cite{magnetic_force} mezi dvěma momenty $\vec{m}_1$ a $\vec{m}_2$, které jsou od sebe vzdáleny $\vec{r}$ \footnote{$\vec{r} = P_1 - P_2$}:

\begin{equation}
    \label{eq:F_m}
    F_m (r,m_1,m_2) = \frac{3\mu_0}{4\pi ||r||^5} 
    \bigg[
        (m_1\cdot r) m_2 +
        (m_2\cdot r) m_1 +
        (m_1\cdot m_2) r -
        \frac{5(m_1\cdot r)(m_2\cdot r)}{||r||^2} r
    \bigg] \\
\end{equation}

2. Magnetické interakce, neboli působení momentu síly \cite{magnetic_torque} na $\vec{m}_2$ z důvodu vytvoření magnetické indukce $B(r, m_1)$ momentem $\vec{m}_1$ \cite{magnetic_force}:

\begin{equation}
    \label{eq:B}
    \begin{split}
        B (r, m) = \frac{\mu_0}{4\pi}\frac{3 \hat{r}(\hat{r}\cdot m) - m}{|r|^3} \\
        \tau = m_2 \times B(r, m_1)
    \end{split}
\end{equation}


%% literatura
\begin{thebibliography}{99}
    \bibitem{magnetic_force} YUNG, Kar W.; LANDECKER, Peter B. a VILLANI, Daniel D. An Analytic Solution for the Force Between Two Magnetic Dipoles. [Online]. \textit{Magnetic and Electrical Separation.} 1998, roč. 9, č. 1, s. 39-52. ISSN 1055-6915. Dostupné z: \url{https://doi.org/10.1155/1998/79537}. [cit. 2023-12-16].
    \bibitem{magnetic_torque} CULLITY, B. D. a GRAHAM, C. D. \textit{Introduction to magnetic materials.} Second edition. Hoboken: IEEE Press, [2009]. ISBN 978-0-471-47741-9.
    \bibitem{torque_def} SERWAY, Raymond A. a JEWETT, John W. Jr. \textit{Physics for scientists and engineers.} 6th ed. Belmont: Thomson-Brooks/Cole, 2004. ISBN 0-534-40842-7.
    % https://www.physicsforums.com/threads/modeling-a-permanent-magnetic-as-a-dipole.594949/#google_vignette
    % https://hal.science/hal-02905104/document
    \bibitem{magnet_dipole_approx} YE, Jianhe; ZHAN, Pengfei; ZENG, Jincheng; KUANG, Honglin; DENG, Yongfang et al. Concise magnetic force model for Halbach-type magnet arrays and its application in permanent magnetic guideway optimization. [Online.] \textit{Journal of Magnetism and Magnetic Materials.} 2023, roč. 587. ISSN 03048853. Dostupné z: \url{https://doi.org/10.1016/j.jmmm.2023.171301}. [cit. 2023-12-16].
    \bibitem{shorthand_error_notation} NATIONAL INSTITUTE OF STANDARDS AND TECHNOLOGY. \textit{Standard Uncertainty and Relative Standard Uncertainty} [online]. [cit. 2023-12-16]. Dostupné z: \url{https://physics.nist.gov/cgi-bin/cuu/Info/Constants/definitions.html}
    \bibitem{shorthand_error_notation_stack_exchange} Jasper. \textit{Shorthand error notation (with brackets) accros decimal point [duplicate]} [online]. [cit. 2023-12-16]. Dostupné z: \url{https://physics.stackexchange.com/questions/445141/shorthand-error-notation-with-brackets-accros-decimal-point}
    \bibitem{tmf_tasks} \textit{Turnaj mladých fyziků} [online]. [cit. 2023-12-16]. Dostupné z: \url{https://tmf.fzu.cz/tasks.php?y}
    \bibitem{magnet_grades} \textit{Neodymium Magnet Grades} [online]. [cit. 2023-12-16]. Dostupné z: \url{https://totalelement.com/blogs/about-neodymium-magnets/neodymium-rare-earth-magnet-grades}
\end{thebibliography}

%% obrázky 
\listoffigures

%% tabulky
\listoftables

\chapter*{Přílohy}
\begin{enumerate}[topsep=0pt, partopsep=0pt]
    \setlength{\itemsep}{0pt}%
    \setlength{\parskip}{0pt}%
    \item \label{attachment_1} Videozáznam interakce nehybného a pomalého spinneru \dotfill \textbf{VID1}
    \item \label{attachment_2} Videozáznam interakce nehybného a rychláho spinneru \dotfill \textbf{VID2}
    \item \label{attachment_3} Videozáznam interakce 3 spinnerů \dotfill \textbf{VID3}
\end{enumerate}
\end{document}