
\chapter{Úvod}
\label{chap:introduction}
Motivací pro tuto práci byla úloha mezinárodní fyzikální soutěže zvané \textit{"International Young Physicists Tournament"}, neboli IYPT.
U nás je však tato soutěž známější pod zkratkou TMF vycházející z překladu původního názvu - \textit{"Turnaj mladých fyziků"}.
Soutěž se v České republice koná pod záštitou Fakulty jaderné a fyzikálně inženýrské ČVUT, FZU AV ČR, MŠMT a JČMT. Podstatou soutěže je dovést středoškolské studenty k vědeckému sledování nejrůznějších jevů ze všech oblastí fyziky.

Úloha, kterou jsme se zabývali, je v pořadí desáta úloha letošního, tedy 37., ročníku. Zadání je následovné \cite{tmf_tasks}:

\textbf{10. Magnetický převod}

\textit{"Vezměte několik identických prstových točítek \footnote{Z překladu anglického "Fidget spinner".} a připevněte k jejich koncům neodymové magnety. Pokud umístíte točítka v rovině vedle sebe a točíte jedním z nich, ostatní se začnou otáčet jen vlivem magnetického pole. Prozkoumejte a vysvětlete tento jev."}

Zadání úlohy je, pro TMF tradičně, velmi otevřené a je tedy pouze na řešiteli, aby si vymezil přesný rozsah svého zkoumání.
Tato práce se bude zabývat:

\begin{enumerate}[topsep=0pt, partopsep=0pt]
    \setlength{\itemsep}{0pt}%
    \setlength{\parskip}{0pt}%
    \item určením vlastností \textit{prstových točítek} (dále "fidget spinner" či pouze "spinner"),
    \item popisem třecích sil působících na spinner,
    \item vývojem simulace chování systémů více fidget spinnerů a porovnáním této simulace s realitou,
    \item přenosem úhlové rychlosti,
    \item přenosem momentu síly,
    \item možným využím získaných poznatků k vývoji efektivnějších magnetických převodů,
    \item vývojem a zkoumáním mnou navržených magnetických převodovek.
\end{enumerate}

\clearpage

\section[Nomenklatura]{Nomenklatura}
\label{sec:nomenclature}
V tabulce \ref{tab:nomenclature} definujeme základní symboly, které budeme používat v průběhu celé práce, společně s jejich významem:

\begin{table}[!ht]
    \captionsetup{justification=raggedright,singlelinecheck=off}
    \caption{Nomenklatura}
    \label{tab:nomenclature}

    \centering{\textbf{Vlastnosti spinneru}} \\
    \begin{tabularx}{\textwidth}{m{0.1\textwidth} m{0.1\textwidth} p{0.4\textwidth} p{5cm} }
        \textbf{Symbol}  & \textbf{Jednotka}         & \textbf{Popis}                                                & \textbf{Poznámka}                              \\
        \hline
        $n$              &                           & Celkový počet ramen                                           & Ekvivaletní počtu magnetů                      \\
        $r$              & $m$                       & Poloměr spinneru                                              & Ekvivaletní vzdálenosti osy otáčení od magnetů \\
        $S$              & $(m,m,0)$                 & Střed spinneru v rovině                                       & (tzn. pozice osy otáčení)                      \\
        $P(i)$           & $(m,m,0)$                 & Pozice $i$. magnetu spinneru                                  & Funkce indexu magnetu                          \\
        $\varphi$        & $\text{rad}$              & Úhel rotace spinneru                                                                                           \\
        $\omega$         & $\text{rad} \cdot s^{-1}$ & Úhlová rychlost spinneru                                                                                       \\
        $I$              & $kg \cdot m^2$            & Moment setrvačnosti spinneru                                                                                   \\
        $\alpha$         & $\text{rad} \cdot s^{-2}$ & Koeficient rychlostně nezávislého brzdného úhlového zrychlení & viz \autoref{chap:drag}                        \\
        $\beta$          & $s^{-1}$                  & Koeficient lineárně závislého brzdného úhlového zrychlení     & viz \autoref{chap:drag}                        \\
        $\gamma$         & $\text{rad}^{-1}$         & Koeficient kvadraticky závislého brzdného úhlového zrychlení  & viz \autoref{chap:drag}                        \\
        $c_1$, $t_{max}$ & $s$                       & Celková doba točení spinneru                                & viz \autoref{chap:drag}                        \\
    \end{tabularx}

    \vspace{24pt}

    \centering{\textbf{Vlastnosti magnetu}} \\
    \begin{tabularx}{\textwidth}{m{0.15\textwidth} m{0.1\textwidth} p{0.45\textwidth} p{4cm} }
        \textbf{Symbol}                      & \textbf{Jednotka}                                                            & \textbf{Popis}                                                                              & \textbf{Poznámka}                      \\
        \hline
        $\vec{m}$                            & $A \cdot m^2$                                                                & Magnetický moment                                                                                                                    \\
        $\vec{B}_r$                          & $T$                                                                          & Remanentní magnetizace                                                                      & (tzv. remanence)                       \\
        $V$                                  & $m^3$                                                                        & Objem magnetu                                                                                                                        \\
        $F_m(\vec{r}, \vec{m_1}, \vec{m_2})$ & $N$                                                                          & Silová interakce mezi magnetickými momenty $\vec{m_1}$ a $\vec{m_2}$ vzdálenými o $\vec{r}$ & \cite{magnetic_force}                  \\
        $B(\vec{r}, \vec{m})$                & $T$                                                                          & Magnetická indukce tvořená magnetickým momentem $\vec{m}$ ve vzdálenosti $\vec{r}$         & \cite{magnetic_force, magnetic_torque} \\
        $\tau_F, \tau_{mag}$                 & $Nm$
                                             & Momenty sil působící na spinner vycházející ze silové a magnetické interakce & \cite{magnetic_torque,torque_def}, viz \autoref{eq:sim_equations2}                                                                   \\
    \end{tabularx}
\end{table}

Dále stojí za zmínku, že pro vyjádření chyby měření může být v textu použita tzv. \textit{shorthand error notation} \cite{shorthand_error_notation} (pro naše účely zkráceno na SEN). Pro jasnost uvedeme příklad, kde zápis pomocí SEN vypadá takto: $11.5(12)$, a ekvivalentní přepis do standardní notace je: $11.5 \pm 1.2$. Tímto zkrátíme zápis $11.5 \pm 1.2$ na $11.5(12)$ \cite{shorthand_error_notation_stack_exchange}.